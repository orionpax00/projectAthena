\chapter[Customizing Tufte-LaTeX]{Customizing \TL}
\label{ch:customizing}

The \TL document classes are designed to closely emulate Tufte's book design by default. However, each document is different and you may encounter situations where the default settings are insufficient. This chapter explores many of the ways you can adjust the \TL document classes to better fit your needs.

%------------------------------------------------

\section{File Hooks}
\label{sec:filehooks}

\index{file hooks|(}
If you create many documents using the \TL classes, it's easier to store your customizations in a separate file instead of copying them into the preamble of each document. The \TL classes provide three file hooks: \docfilehook{tufte-common-local.tex}{common}, \docfilehook{tufte-book-local.tex}{book}, and \docfilehook{tufte-handout-local.tex}{handout}.\sloppy

\begin{description}
\item[\docfilehook{tufte-common-local.tex}{common}]
If this file exists, it will be loaded by all of the \TL document classes just prior to any document-class-specific code. If your customizations or code should be included in both the book and handout classes, use this file hook.
\item[\docfilehook{tufte-book-local.tex}{book}] 
If this file exists, it will be loaded after all of the common and book-specific code has been read. If your customizations apply only to the book class, use this file hook.
\item[\docfilehook{tufte-common-handout.tex}{handout}] 
If this file exists, it will be loaded after all of the common and handout-specific code has been read. If your customizations apply only to the handout class, use this file hook.
\end{description}

\index{file hooks|)}

%------------------------------------------------

\section{Numbered Section Headings}
\label{sec:numbered-sections}
\index{headings!numbered}

While Tufte dispenses with numbered headings in his books, if you require them, they can be enabled by changing the value of the \doccounter{secnumdepth} counter. From the table below, select the heading level at which numbering should stop and set the \doccounter{secnumdepth} counter to that value. For example, if you want parts and chapters numbered, but don't want numbering for sections or subsections, use the command:
\begin{docspec}
\doccmd{setcounter}\{secnumdepth\}\{0\}
\end{docspec}

The default \doccounter{secnumdepth} for the \TL document classes is $-1$.

\begin{table}
\footnotesize
\begin{center}
\begin{tabular}{lr}
\toprule
Heading level & Value \\
\midrule
Part (in \doccls{tufte-book}) & $-1$ \\
Part (in \doccls{tufte-handout}) & $0$ \\
Chapter (only in \doccls{tufte-book}) & $0$ \\
Section & $1$ \\
Subsection & $2$ \\
Subsubsection & $3$ \\
Paragraph & $4$ \\
Subparagraph & $5$ \\
\bottomrule
\end{tabular}
\end{center}
\caption{Heading levels used with the \texttt{secnumdepth} counter.}
\end{table}

%------------------------------------------------

\section{Changing the Paper Size}
\label{sec:paper-size}

The \TL classes currently only provide three paper sizes: \textsc{a4}, \textsc{b5}, and \textsc{us} letter. To specify a different paper size (and/or margins), use the \doccmd[geometry]{geometrysetup} command in the preamble of your document (or one of the file hooks). The full documentation of the \doccmd{geometrysetup} command may be found in the \docpkg{geometry} package documentation.\cite{pkg-geometry}

%------------------------------------------------

\section{Customizing Marginal Material}
\label{sec:marginal-material}

Marginal material includes sidenotes, citations, margin notes, and captions. Normally, the justification of the marginal material follows the justification of the body text. If you specify the \docclsopt{justified} document class option, all of the margin material will be fully justified as well. If you don't specify the \docclsopt{justified} option, then the marginal material will be set ragged right.

You can set the justification of the marginal material separately from the body text using the following document class options: \docclsopt{sidenote}, \docclsopt{marginnote}, \docclsopt{caption}, \docclsopt{citation}, and \docclsopt{marginals}. Each option refers to its obviously corresponding marginal material type. The \docclsopt{marginals} option simultaneously sets the justification on all four marginal material types.

Each of the document class options takes one of five justification types:
\begin{description}
\item[\docclsopt{justified}] Fully justifies the text (sets it flush left and right).
\item[\docclsopt{raggedleft}] Sets the text ragged left, regardless of which page it falls on.
\item[\docclsopt{raggedright}] Sets the text ragged right, regardless of which page it falls on.
\item[\doccls{raggedouter}] Sets the text ragged left if it falls on the left-hand (verso) page of the spread and otherwise sets it ragged right. This is useful in conjunction with the \docclsopt{symmetric} document class option.
\item[\docclsopt{auto}] If the \docclsopt{justified} document class option was specified, then set the text fully justified; otherwise the text is set ragged right. This is the default justification option if one is not explicitly specified.
\end{description}

\noindent For example, 
\begin{docspec}
\doccmdnoindex{documentclass}[symmetric,justified,marginals=raggedouter]\{tufte-book\}
\end{docspec}
will set the body text of the document to be fully justified and all of the margin material (sidenotes, margin notes, captions, and citations) to be flush against the body text with ragged outer edges.

\newthought{The font and style} of the marginal material may also be modified using the following commands:

\begin{docspec}
\doccmd{setsidenotefont}\{\docopt{font commands}\}\\
\doccmd{setcaptionfont}\{\docopt{font commands}\}\\
\doccmd{setmarginnotefont}\{\docopt{font commands}\}\\
\doccmd{setcitationfont}\{\docopt{font commands}\}
\end{docspec}

The \doccmddef{setsidenotefont} sets the font and style for sidenotes, the \doccmddef{setcaptionfont} for captions, the \doccmddef{setmarginnotefont} for margin notes, and the \doccmddef{setcitationfont} for citations. The \docopt{font commands} can contain font size changes (e.g., \doccmdnoindex{footnotesize}, \doccmdnoindex{Huge}, etc.), font style changes (e.g., \doccmdnoindex{sffamily}, \doccmdnoindex{ttfamily}, \doccmdnoindex{itshape}, etc.), color changes (e.g., \doccmdnoindex{color}\texttt{\{blue\}}), and many other adjustments.

If, for example, you wanted the captions to be set in italic sans serif, you could use:
\begin{docspec}
\doccmd{setcaptionfont}\{\doccmdnoindex{itshape}\doccmdnoindex{sffamily}\}
\end{docspec}
